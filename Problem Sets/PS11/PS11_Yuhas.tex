\documentclass{article}
\usepackage[utf8]{inputenc}

\title{PS11 Yuhas}
\author{myuhas5120 }
\date{April 2018}

\usepackage[authoryear]{natbib}

\begin{document}

\maketitle

\section{Introduction}
\begin{itemize}
    \item Type of craft beer predicted by ABV of Beer
    \item Potentially increase accuracy by sub-setting by brewery.
\end{itemize}

\section{Literature Review}
\begin{itemize}
    \item Craft beer rising share of beverages markets
    \item Craft beer associated with niche knowledge
    \item Breweries recently creating stronger beers
\end{itemize}

\section{Data}
\begin{itemize}
    \item obtained from Kaggle 
at:"https://www.kaggle.com/nickhould/craft-cans/data"
    \item 64 of 2411 have no ABV(Alcohol By Volume) values and are 
dropped.
    \item since some breweries are missing ABV values for all drinks, 
subsetting by brewery is not currently an option.
    \item Potential future fix is to manually find ABV values for each 
of the 64 beers.
\end{itemize}

\section{Methods}
\begin{itemize}
    \item k Nearest Neighbors(kNN): Six fold Cross Validation with 15 
random guess tuning
    \item Support Vector Machine(SVM): Six fold Cross Validation with 15 
random guess tuning
    \item I expect SVM to provide a better model due to the numerous 
types of beer and possible ABV values.
\end{itemize}

\section{Findings}
\begin{itemize}
    \item I expect the model to have moderate accuracy at best due to 
the smaller range of ABV values to beer types.
\end{itemize}

\section{Conclusion}
\begin{itemize}
    \item In conclusion, I believe that ABV may be easier to predict 
based on type of beer which is a reversal of this project; however, the 
possibilities provided by this analysis may be valuable in public 
education of craft brewing.
\end{itemize}

\bibliographystyle{jpe}
\nocite{*}
\bibliography{PS11_Yuhas.bib}


\end{document}

